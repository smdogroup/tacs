\documentclass{article}
\usepackage[linkcolor=black,colorlinks]{hyperref}
\usepackage{amsmath}
\usepackage{setspace}
\usepackage[margin=1.25in]{geometry}

\title{An installation guide for TACS}
\author{Graeme Kennedy}

\begin{document}

\maketitle

\section{Introduction}

The Toolbox for the Analysis of Composite Structures (TACS) is a
parallel finite-element code written in C++, with an optional python
interface. The architecture of TACS is designed with gradient-based
optimization in mind. There are built-in routines for evaluating
functions of interest and their sensitivities. TACS is object oriented
and can be extended to include new elements, constitutive properties
or functions of interest.

There are several software package dependencies required to install
TACS. The dependencies are divided into the following categories:
\begin{enumerate}
\item Dependencies for the C++ interface
\item Dependencies for the python interface
\end{enumerate}

\section{Checking out the code}

The TACS source is located at:
\url{https://bitbucket.org/mdolab/tacs}. Using mercurial, checkout the
source by typing \texttt{hg clone
  https://user@bitbucket.org/mdolab/tacs}

Also, download the file: \texttt{TACS\_extern.tar.gz} and place it in
the directory \texttt{TACS/extern}. Extract it and follow the
instructions in the README file in this directory. Note also that
\texttt{f5totec} is located in this directory as well.

The first step after checking out TACS is to copy
\texttt{Makefile.in.info} to \texttt{Makefile.in} in the TACS root
directory. \texttt{Makefile.in} contains the installation paths of the
libraries used by TACS. In particular, set \texttt{TACS\_DIR} to the
root directory of the TACS installation.

\section{Installing the C++ interface}

TACS can be used as a stand-alone C++ code without the python-level
interface. The C++ source is located under \texttt{TACS/src}. 

The requirements for the installation are:
%
\begin{enumerate}
\item An MPI installation implementing the MPI 2.0 Standard
  \url{http://www.mcs.anl.gov/research/projects/mpi/}
\item \texttt{LAPACK}: The standard linear algebra package
\item \texttt{METIS 4.0}: Serial graph-partitioner
  \url{http://glaros.dtc.umn.edu/gkhome/views/metis}
\item \texttt{f5totec}: A utility for converting .f5 files generated
  by TACS to a tecplot compatible format
\item \texttt{TecIO}: Tecplot binary format conversion code
\item \texttt{AMD}: A package for computing the minimum degree
  ordering \url{http://www.cise.ufl.edu/research/sparse/amd/}
  \begin{itemize}
  \item \texttt{UFconfig}: Required for the \texttt{AMD} package:
    \url{http://www.cise.ufl.edu/research/sparse/UFconfig/}
  \end{itemize}
\end{enumerate}

The external programs: \texttt{METIS}, \texttt{AMD}, \texttt{UFconfig}
and \texttt{TecIO} are part of \texttt{TACS\_extern.tar.gz} file.

Notes: An MPI 2.0 standard compliant compiler is essential. If one is
not available, the code will not compile correctly. When you
think everything is ready, type \texttt{make} in the TACS
directory. This makes all the TACS-only source.

\subsection{f5totec}

\texttt{f5totec} is a conversion utility created by Graeme Kennedy for
the conversion of .f5 files output from TACS to a tecplot compatible
format. I recommend putting \texttt{f5totec} on your path. One way is
to create a symbolic link in your local \texttt{bin} directory to
f5totec. Note that this utility converts one .f5 file at a time. The
usage is \texttt{f5totec input [output]}, if no output file is
specified, the output name is taken as the input name with the
extension .plt.

\subsection{Tutorial example}

There is a simple tutorial example in \texttt{TACS/tutorial}. Once the
C++ interface has been compiled, try making and running
\texttt{plate\_example.c}.

\section{Installing the python interface}

The python interface is generated in the \texttt{TACS/python}
directory.  The interface is generated using program \texttt{SWIG}
which can be obtained here: \url{http://www.swig.org/}. Make sure you
use version \texttt{SWIG 1.3.40} or greater.

The python interface requires the following packages:
%
\begin{enumerate}
\item \texttt{numpy}: Numerical python packages
\item \texttt{mpi4py}: Python interface for MPI
\end{enumerate}

You will need a recent version of python - at least python2.5 or
greater (I have not tested with python 3.0). It is possible to install
a local version of python if an old version is installed.

Specify the location of the python include files using
\texttt{PYTHON\_DIR}.  If you're using the native version then the
director is usually: \texttt{PYTHON\_DIR=/usr/include/python2.x}. If
you're using a locally installed version, point this path to the local
header files.

Once the packages have been installed, and you have a recent version
of SWIG, the python interface can be installed by entering
\texttt{TACS/python} and typing make.

\subsection{NumPy}

NumPy is available from here \url{http://numpy.scipy.org/}. Make sure
to specify the location of the numpy include files in
\texttt{Makefile.in}. This can be found by starting python at the command line
then typing: \texttt{import numpy; numpy.get\_include()}.

\subsection{mpi4py}
 
\texttt{mpi4py} can be downloaded from here:
\url{http://code.google.com/p/mpi4py/}.  Install the package using:
\texttt{python setup.py build} then \texttt{python setup.py
  install}. If you don't have root access use either the
\texttt{prefix} or the \texttt{home} options to specify the local
installation location. The line in the \texttt{Makefile.in} can be
found by starting python at the command line and typing \texttt{import
  mpi4py; mpi4py.get\_include()}.

\section{Installing TriPan and its python interface}

TriPan is a full three-dimensional panel code that is tightly
integrated with TACS. Running TriPan requires the installation of
PETSc: \url{http://www.mcs.anl.gov/petsc/petsc-as/}.
TriPan works with PETSc 3.2. Earlier versions of PETSc are incompatible
with TriPan due to various modifications that have been made to the
PETSc API.

I recommend installing PETSc with the options:

\texttt{./configure --with-pic --with-cc=mpicc --with-fc=mpif90
  --with-cxx=mpicxx --with -debugging=0 --with-shared-libraries}

After installing PETSc, you should be able to compile TriPan and the
python interface to TriPan in \texttt{tripan/python}.

\end{document}
